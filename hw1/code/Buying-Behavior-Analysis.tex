% Options for packages loaded elsewhere
\PassOptionsToPackage{unicode}{hyperref}
\PassOptionsToPackage{hyphens}{url}
\PassOptionsToPackage{dvipsnames,svgnames,x11names}{xcolor}
%
\documentclass[
]{article}
\usepackage{amsmath,amssymb}
\usepackage{iftex}
\ifPDFTeX
  \usepackage[T1]{fontenc}
  \usepackage[utf8]{inputenc}
  \usepackage{textcomp} % provide euro and other symbols
\else % if luatex or xetex
  \usepackage{unicode-math} % this also loads fontspec
  \defaultfontfeatures{Scale=MatchLowercase}
  \defaultfontfeatures[\rmfamily]{Ligatures=TeX,Scale=1}
\fi
\usepackage{lmodern}
\ifPDFTeX\else
  % xetex/luatex font selection
\fi
% Use upquote if available, for straight quotes in verbatim environments
\IfFileExists{upquote.sty}{\usepackage{upquote}}{}
\IfFileExists{microtype.sty}{% use microtype if available
  \usepackage[]{microtype}
  \UseMicrotypeSet[protrusion]{basicmath} % disable protrusion for tt fonts
}{}
\makeatletter
\@ifundefined{KOMAClassName}{% if non-KOMA class
  \IfFileExists{parskip.sty}{%
    \usepackage{parskip}
  }{% else
    \setlength{\parindent}{0pt}
    \setlength{\parskip}{6pt plus 2pt minus 1pt}}
}{% if KOMA class
  \KOMAoptions{parskip=half}}
\makeatother
\usepackage{xcolor}
\usepackage[margin=1in]{geometry}
\usepackage{color}
\usepackage{fancyvrb}
\newcommand{\VerbBar}{|}
\newcommand{\VERB}{\Verb[commandchars=\\\{\}]}
\DefineVerbatimEnvironment{Highlighting}{Verbatim}{commandchars=\\\{\}}
% Add ',fontsize=\small' for more characters per line
\usepackage{framed}
\definecolor{shadecolor}{RGB}{248,248,248}
\newenvironment{Shaded}{\begin{snugshade}}{\end{snugshade}}
\newcommand{\AlertTok}[1]{\textcolor[rgb]{0.94,0.16,0.16}{#1}}
\newcommand{\AnnotationTok}[1]{\textcolor[rgb]{0.56,0.35,0.01}{\textbf{\textit{#1}}}}
\newcommand{\AttributeTok}[1]{\textcolor[rgb]{0.13,0.29,0.53}{#1}}
\newcommand{\BaseNTok}[1]{\textcolor[rgb]{0.00,0.00,0.81}{#1}}
\newcommand{\BuiltInTok}[1]{#1}
\newcommand{\CharTok}[1]{\textcolor[rgb]{0.31,0.60,0.02}{#1}}
\newcommand{\CommentTok}[1]{\textcolor[rgb]{0.56,0.35,0.01}{\textit{#1}}}
\newcommand{\CommentVarTok}[1]{\textcolor[rgb]{0.56,0.35,0.01}{\textbf{\textit{#1}}}}
\newcommand{\ConstantTok}[1]{\textcolor[rgb]{0.56,0.35,0.01}{#1}}
\newcommand{\ControlFlowTok}[1]{\textcolor[rgb]{0.13,0.29,0.53}{\textbf{#1}}}
\newcommand{\DataTypeTok}[1]{\textcolor[rgb]{0.13,0.29,0.53}{#1}}
\newcommand{\DecValTok}[1]{\textcolor[rgb]{0.00,0.00,0.81}{#1}}
\newcommand{\DocumentationTok}[1]{\textcolor[rgb]{0.56,0.35,0.01}{\textbf{\textit{#1}}}}
\newcommand{\ErrorTok}[1]{\textcolor[rgb]{0.64,0.00,0.00}{\textbf{#1}}}
\newcommand{\ExtensionTok}[1]{#1}
\newcommand{\FloatTok}[1]{\textcolor[rgb]{0.00,0.00,0.81}{#1}}
\newcommand{\FunctionTok}[1]{\textcolor[rgb]{0.13,0.29,0.53}{\textbf{#1}}}
\newcommand{\ImportTok}[1]{#1}
\newcommand{\InformationTok}[1]{\textcolor[rgb]{0.56,0.35,0.01}{\textbf{\textit{#1}}}}
\newcommand{\KeywordTok}[1]{\textcolor[rgb]{0.13,0.29,0.53}{\textbf{#1}}}
\newcommand{\NormalTok}[1]{#1}
\newcommand{\OperatorTok}[1]{\textcolor[rgb]{0.81,0.36,0.00}{\textbf{#1}}}
\newcommand{\OtherTok}[1]{\textcolor[rgb]{0.56,0.35,0.01}{#1}}
\newcommand{\PreprocessorTok}[1]{\textcolor[rgb]{0.56,0.35,0.01}{\textit{#1}}}
\newcommand{\RegionMarkerTok}[1]{#1}
\newcommand{\SpecialCharTok}[1]{\textcolor[rgb]{0.81,0.36,0.00}{\textbf{#1}}}
\newcommand{\SpecialStringTok}[1]{\textcolor[rgb]{0.31,0.60,0.02}{#1}}
\newcommand{\StringTok}[1]{\textcolor[rgb]{0.31,0.60,0.02}{#1}}
\newcommand{\VariableTok}[1]{\textcolor[rgb]{0.00,0.00,0.00}{#1}}
\newcommand{\VerbatimStringTok}[1]{\textcolor[rgb]{0.31,0.60,0.02}{#1}}
\newcommand{\WarningTok}[1]{\textcolor[rgb]{0.56,0.35,0.01}{\textbf{\textit{#1}}}}
\usepackage{graphicx}
\makeatletter
\def\maxwidth{\ifdim\Gin@nat@width>\linewidth\linewidth\else\Gin@nat@width\fi}
\def\maxheight{\ifdim\Gin@nat@height>\textheight\textheight\else\Gin@nat@height\fi}
\makeatother
% Scale images if necessary, so that they will not overflow the page
% margins by default, and it is still possible to overwrite the defaults
% using explicit options in \includegraphics[width, height, ...]{}
\setkeys{Gin}{width=\maxwidth,height=\maxheight,keepaspectratio}
% Set default figure placement to htbp
\makeatletter
\def\fps@figure{htbp}
\makeatother
\setlength{\emergencystretch}{3em} % prevent overfull lines
\providecommand{\tightlist}{%
  \setlength{\itemsep}{0pt}\setlength{\parskip}{0pt}}
\setcounter{secnumdepth}{5}
\ifLuaTeX
  \usepackage{selnolig}  % disable illegal ligatures
\fi
\usepackage{bookmark}
\IfFileExists{xurl.sty}{\usepackage{xurl}}{} % add URL line breaks if available
\urlstyle{same}
\hypersetup{
  pdftitle={Analysis of Household Buying Behavior: Carbonated Soft Drinks and Other Beverages},
  pdfauthor={Giovanni Compiani},
  colorlinks=true,
  linkcolor={Maroon},
  filecolor={Maroon},
  citecolor={Blue},
  urlcolor={blue},
  pdfcreator={LaTeX via pandoc}}

\title{Analysis of Household Buying Behavior: Carbonated Soft Drinks and
Other Beverages}
\author{Giovanni Compiani}
\date{}

\begin{document}
\maketitle

{
\hypersetup{linkcolor=}
\setcounter{tocdepth}{2}
\tableofcontents
}
\setlength{\parskip}{6pt}
\newpage

\section{Overview}\label{overview}

We will perform an analysis of category buying and consumption behavior
of beverages (not including alcohol or dairy), with a particular
emphasis on sodas (carbonated soft drinks/CSD's). Sodas have been the
subject of an intense debate linked to public health concerns, and one
or two cent-per-ounce taxes have recently been passed in San Francisco,
Cook County (although the tax was repealed in 2017), Philadelphia,
Boulder, and other cities. Changes in buying behavior due to health
concerns present challenges to the beverage manufacturers, but also
opportunities if consumers are shifting their consumption to healthier
substitutes.

\section{Data}\label{data}

The Nielsen Homescan panel data set is an ideal data source to study
broad trends in consumption behavior. This data set is available for
research and teaching purposes from the Kilts Center for Marketing at
Chicago Booth.

We will use Homescan panel data from 2004 to 2014. In many years we have
information on the buying behavior of more than sixty thousand
households. The corresponding full data set is large. Hence, to avoid
memory and computing time issues, I extracted the beverage data that we
will use for the analysis. I also took a \emph{25 percent random
subsample} of all the original obervations. Once you load the data, you
can verify that even this subsample of the beverage data contains more
than 10 million observations (use \texttt{nrow}, \texttt{ncol}, or
\texttt{dim} to see the size of a data.table or data frame).

Although not necessary for this assignment, it is good to know how to
create random subsamples. For more information, please consult the
Appendix below.

The key data for the analysis are contained in a \textbf{purchase file}
and a \textbf{product file}. Load the data:

\begin{Shaded}
\begin{Highlighting}[]
\FunctionTok{library}\NormalTok{(bit64)}
\FunctionTok{library}\NormalTok{(data.table)}

\NormalTok{data\_folder    }\OtherTok{=} \StringTok{"/Users/nichada/MyCode/MPCS/BUSN\_DataSci"}
\NormalTok{proj\_folder }\OtherTok{=} \StringTok{"/DataSci\_Mkt/data"}
\NormalTok{purchases\_file }\OtherTok{=} \StringTok{"purchases\_beverages.RData"}
\NormalTok{products\_file  }\OtherTok{=} \StringTok{"products\_beverages.RData"}

\FunctionTok{load}\NormalTok{(}\FunctionTok{paste0}\NormalTok{(data\_folder, }\StringTok{"/"}\NormalTok{, proj\_folder, }\StringTok{"/"}\NormalTok{, purchases\_file))}
\FunctionTok{load}\NormalTok{(}\FunctionTok{paste0}\NormalTok{(data\_folder, }\StringTok{"/"}\NormalTok{, proj\_folder, }\StringTok{"/"}\NormalTok{, products\_file))}
\end{Highlighting}
\end{Shaded}

Note that the string \texttt{data\_folder} contains the location of the
data, i.e.~the data are \emph{assumed} to be in that folder! The
\texttt{paste0} command merges strings. The bit64 package is used
because the product UPC numbers are 64-bit (long) integers, a data type
that is not part of base R.

\newpage

\section{Variable description}\label{variable-description}

The \textbf{\texttt{purchases} data.table} contains information on the
products bought by the households.

\begin{itemize}
\tightlist
\item
  \textbf{Inspect the data}
\end{itemize}

\begin{Shaded}
\begin{Highlighting}[]
\CommentTok{\# Inspect structure of the data}
\FunctionTok{str}\NormalTok{(purchases)}

\CommentTok{\# View first few rows of the data}
\FunctionTok{head}\NormalTok{(purchases)}
\end{Highlighting}
\end{Shaded}

Households are identified based on a unique \texttt{household\_code}.
For each shopping trip we know the \texttt{purchase\_date} and the
\texttt{retailer\_code} (for confidentiality, the Kilts Center data do
not include the exact name of the retailer).

For each product we have information on the \texttt{total\_price\_paid}
and the \texttt{quantity} purchased. A deal flag (0/1) and a
\texttt{coupon\_value} is also provided. The \texttt{total\_price\_paid}
applies to \emph{all} units purchased. Hence, if
\texttt{total\_price\_paid\ =\ 7.98} and \texttt{quantity\ =\ 2}, then
the \emph{per-unit price} is 7.98/2 = 3.99 dollars. Furthermore, if the
\texttt{coupon\_value} is positive, then the total dollar amount that
the household spent is \texttt{total\_price\_paid\ -\ coupon\_value}.
The \emph{per-unit cost} to the household is then even lower. For
example, if \texttt{coupon\_value\ =\ 3} in the example above, then the
per-unit cost to the household is (7.98 - 3)/2 = 2.49 dollars.

I recommend to use the per-unit cost if the objective is to measure
household dollar spending per unit purchased. If the objective is to
measure the shelf-price of the product, use the per-unit price instead.

\bigskip

\textbf{Important data notes}

\begin{enumerate}
\def\labelenumi{\arabic{enumi}.}
\item
  Products in the Nielsen Homescan and RMS scanner data are identified
  by a unique combination of \texttt{upc} and \texttt{upc\_ver\_uc}. Why
  not just the UPC code? --- Because UPC's can change over time, for
  example if a UPC is assigned to a different product. The UPC version
  code captures such a change. From now on, whenever we refer to a
  \emph{product}, we mean a \texttt{upc}/\texttt{upc\_ver\_uc}
  combination. To identify a unique product in data.table, use a
  \texttt{by\ =\ .(upc,\ upc\_ver\_uc)} statement.
\item
  The \texttt{panel\_year} variable in \texttt{purchases} is intended
  \emph{only} to link households to the corresponding household
  attribute data (income, age, etc.), which are updated yearly. At the
  beginning of a \texttt{panel\_year} it need not exactly correspond to
  the calendar year. We will work with the household attribute data in
  one of the next assignments.
\end{enumerate}

The \textbf{\texttt{products}} data.table contains product information
for each \texttt{upc}/\texttt{upc\_ver\_uc} combination.

\begin{itemize}
\tightlist
\item
  \textbf{Inspect the data.table}
\end{itemize}

\begin{Shaded}
\begin{Highlighting}[]
\CommentTok{\# Inspect structure of the data}
\FunctionTok{str}\NormalTok{(products)}

\CommentTok{\# View first few rows of the data}
\FunctionTok{head}\NormalTok{(products)}
\end{Highlighting}
\end{Shaded}

Note that products are organized into departments (e.g.~DRY GROCERY),
product groups (e.g.~CARBONATED BEVERAGES), and product modules
(e.g.~SOFT DRINKS - CARBONATED), with corresponding codes and
descriptions. Use \texttt{unique} or \texttt{table} to print all values
for the variables. Or create a table that lists all the product module
codes, product module descriptions, and group descriptions in the data:

\begin{Shaded}
\begin{Highlighting}[]
\NormalTok{module\_DT }\OtherTok{=}\NormalTok{ products[, }\FunctionTok{head}\NormalTok{(.SD, }\DecValTok{1}\NormalTok{), by }\OtherTok{=}\NormalTok{ product\_module\_code,}
\NormalTok{                       .SDcols }\OtherTok{=} \FunctionTok{c}\NormalTok{(}\StringTok{"product\_module\_descr"}\NormalTok{, }\StringTok{"product\_group\_descr"}\NormalTok{)]}
\NormalTok{module\_DT[}\FunctionTok{order}\NormalTok{(product\_group\_descr)]}
\end{Highlighting}
\end{Shaded}

We also have brand codes and descriptions, such as PEPSI R (Pepsi
regular). You will often seen the brand description CTL BR, which stands
for \emph{control brand}, i.e.~private label brand. Brands are
identified using either the \texttt{brand\_code\_uc} or
\texttt{brand\_descr} variables, and are sold as different products
(\texttt{upc}/\texttt{upc\_ver\_uc}) that differ along size, form
(e.g.~bottles vs.~cans), or flavor.

\texttt{multi} indicates the number of units in a multi-pack (a
multi-pack is a pack size such that \texttt{multi\ \textgreater{}\ 1}).
More on the amount and unit variables below.

For many more details on the data, consult the \emph{Consumer Panel
Dataset Manual} (on Canvas).

\newpage

\section{Prepare the data for the
analysis}\label{prepare-the-data-for-the-analysis}

We will calculate yearly summary statistics of customer buying behavior.
To calculate the year corresponding to a purchase date, use
\texttt{year()} in the data.table IDateTime class (see
\texttt{?IDateTime} to learn about other conversion functions). Note
that there are other methods for time aggregation that we will study
later in the course.

\begin{Shaded}
\begin{Highlighting}[]
\NormalTok{purchases[, year }\SpecialCharTok{:=} \FunctionTok{year}\NormalTok{(purchase\_date)]}
\FunctionTok{head}\NormalTok{(purchases)}
\end{Highlighting}
\end{Shaded}

Note that we create this year-variable because the \texttt{panel\_year}
variable in \texttt{purchases} does not exactly correspond to the
calendar year, as we already discussed above. You can verify that there
is a tiny percentage of observations for the 2003 calender year, that
``slipped'' into the data set because the purchases are recorded for
households in the 2004 \texttt{panel\_year}.

\begin{itemize}
\tightlist
\item
  \textbf{Remove the 2003 observations}
\end{itemize}

\begin{Shaded}
\begin{Highlighting}[]
\CommentTok{\# Remove observations where the year is 2003}
\NormalTok{purchases }\OtherTok{\textless{}{-}}\NormalTok{ purchases[year }\SpecialCharTok{!=} \DecValTok{2003}\NormalTok{]}

\CommentTok{\# Check if the 2003 observations were successfully removed}
\FunctionTok{summary}\NormalTok{(purchases}\SpecialCharTok{$}\NormalTok{year)}
\end{Highlighting}
\end{Shaded}

\subsection{Define categories}\label{define-categories}

In the analysis we want to distinguish between carbonated soft drinks
(CSD's), diet (low-calorie) CSD's, bottled water, and a category
including all other beverages (juices, \ldots). Hence, we create a new
\texttt{category} variable that allows us classify the beverage purchase
observations.

\begin{itemize}
\tightlist
\item
  \textbf{First, in the \texttt{products} table, create a default
  \texttt{category} variable with a name such as ``Other''. Then find
  the product module codes for the three relevant categories and assign
  a corresponding name to \texttt{category}}.
\item
  \textbf{Document the number of observations, i.e.~the number of
  products that belong to each of the categories}.
\end{itemize}

\begin{Shaded}
\begin{Highlighting}[]
\CommentTok{\# Assuming \textquotesingle{}products\textquotesingle{} is your data.table}

\CommentTok{\# Step 1: Create a default category called \textquotesingle{}Other\textquotesingle{}}
\NormalTok{products[, category }\SpecialCharTok{:=} \StringTok{"Other"}\NormalTok{]}

\CommentTok{\# Step 2: Assign specific categories based on \textquotesingle{}product\_module\_descr\textquotesingle{}}
\NormalTok{products[product\_module\_code }\SpecialCharTok{==} \StringTok{"1484"}\NormalTok{, category }\SpecialCharTok{:=} \StringTok{"Carbonated Soft Drinks"}\NormalTok{]}
\NormalTok{products[product\_module\_code }\SpecialCharTok{==} \StringTok{"1553"}\NormalTok{, category }\SpecialCharTok{:=} \StringTok{"Diet Soft Drinks"}\NormalTok{]}
\NormalTok{products[product\_module\_code }\SpecialCharTok{==} \StringTok{"1487"}\NormalTok{, category }\SpecialCharTok{:=} \StringTok{"Bottled Water"}\NormalTok{]}

\CommentTok{\# Step 3: Check the distribution of categories}
\CommentTok{\# table(products$category)}
\NormalTok{products[, .N, by }\OtherTok{=}\NormalTok{ category]}
\end{Highlighting}
\end{Shaded}

Now merge the category variable with the purchase data.

\begin{Shaded}
\begin{Highlighting}[]
\NormalTok{purchases }\OtherTok{=} \FunctionTok{merge}\NormalTok{(purchases, products[, .(upc, upc\_ver\_uc, category)])}
\end{Highlighting}
\end{Shaded}

To understand why this statement works, inspect the keys of the two
tables (e.g.~\texttt{key(purchases)}). The keys contain two common
columns, \texttt{upc} and \texttt{upc\_ver\_uc}, which are used as the
foreign key that provides the product-level link between the products
and purchase tables.

\subsection{Volume in equivalent
units}\label{volume-in-equivalent-units}

To measure volume in \emph{equivalent units}, we need the product-level
information on the \emph{units of measurement} of product volume
(\texttt{size1\_units}), such as ounces, and the corresponding volume in
a pack size (\texttt{size1\_amount1}). This information neeeds to be
merged with the purchase data. We also merge with \texttt{multi}, which
indicates multi-pack sizes.

\begin{itemize}
\tightlist
\item
  \textbf{Perform this merge}
\end{itemize}

For beverages, product volume is typically measured in ounces
(\texttt{OZ}), less frequently in quarts (\texttt{QT}), and only rarely
in counts (\texttt{CT}).

\begin{itemize}
\tightlist
\item
  \textbf{Document the number of observations by unit of measurement.
  Let's ignore counts, and remove all corresponding data from the
  purchases data.table. Then convert the \texttt{quantity} of units
  purchased into a common \texttt{volume} measure in gallons, using the
  \texttt{size1\_amount} variable. Also incorporate the \texttt{multi}
  variable into the volume calculation---\texttt{multi} accounts for
  multi-packs.}
\end{itemize}

\emph{Hint}: For example, for observations where \texttt{size1\_units}
is \texttt{OZ}, you can calculate the volume:

\begin{quote}
\texttt{volume\ :=\ (1/128)*size1\_amount*multi*quantity}
\end{quote}

\subsection{Number of households in the
data}\label{number-of-households-in-the-data}

To calculate the number of households in the data by year, use:

\begin{Shaded}
\begin{Highlighting}[]
\NormalTok{purchases[, no\_households }\SpecialCharTok{:=} \FunctionTok{length}\NormalTok{(}\FunctionTok{unique}\NormalTok{(household\_code)), by }\OtherTok{=}\NormalTok{ year]}
\end{Highlighting}
\end{Shaded}

\begin{itemize}
\tightlist
\item
  \textbf{Create and show a table with the number of households by year}
\end{itemize}

Note the expansion in the number of Homescan panelists in 2007!

\newpage

\section{Category-level analysis}\label{category-level-analysis}

Now we are ready to analyse the evolution of purchases and consumption
in the four product categories. We want to calculate total and per
capita (more precisely: per household) consumption metrics. First, we
create the total dollar spend and the total purchase volume for each
category/year combination. We use the data.table approach for
aggregation:

\begin{Shaded}
\begin{Highlighting}[]
\NormalTok{purchases\_category }\OtherTok{=}\NormalTok{ purchases[,}
\NormalTok{   .(}\AttributeTok{spend           =} \FunctionTok{sum}\NormalTok{(total\_price\_paid }\SpecialCharTok{{-}}\NormalTok{ coupon\_value),}
     \AttributeTok{purchase\_volume =} \FunctionTok{sum}\NormalTok{(volume),}
     \AttributeTok{no\_households   =} \FunctionTok{head}\NormalTok{(no\_households, }\DecValTok{1}\NormalTok{)),}
\NormalTok{   keyby }\OtherTok{=}\NormalTok{ .(category, year)]}
\end{Highlighting}
\end{Shaded}

\begin{itemize}
\tightlist
\item
  \textbf{Calculate per capita spend and purchase volume (in gallons)
  for each category separately. Then graph the evolution of the yearly
  per capita purchase volume for all four categories}.
\end{itemize}

Note: Instead of creating graphs for each of the four categories you can
use a \texttt{facet\_wrap} layer provided by ggplot2.

\begin{itemize}
\tightlist
\item
  \textbf{Express the purchase/consumption data as multiples of the 2004
  values, such that per capita volume takes the value of 1.0 in all
  categories in 2004. Such a normalization allows us to compare the
  consumption series in each category directly in percentage terms. Then
  show the graphs of consumption (normalized to its 2004 value), and
  discuss the results}.
\end{itemize}

\newpage

\section{Brand-level analysis}\label{brand-level-analysis}

Now we investigate the evolution of consumption for some of the key
brands in the soda and bottled water categories.

\begin{itemize}
\tightlist
\item
  \textbf{First, merge the brand identifier \texttt{brand\_descr} with
  the purchase data}
\end{itemize}

Then we rank brands by total dollar spend in each category separately.
We can assign ranks either using the \texttt{rank} function in base R,
or using \texttt{frankv} (or \texttt{frank}) in the data.table package.
Simple usage: To rank a vector \texttt{x} in ascending order, use
\texttt{frankv(x)}. To rank in descending order, use
\texttt{frankv(x,\ order\ =\ -1)}. See \texttt{?frank} for more options.

First calculate total dollar spend by each category/brand combination,
then assign the rank according to total spend:

\begin{Shaded}
\begin{Highlighting}[]
\NormalTok{brand\_summary }\OtherTok{=}\NormalTok{ purchases[, .(}\AttributeTok{spend =} \FunctionTok{sum}\NormalTok{(total\_price\_paid }\SpecialCharTok{{-}}\NormalTok{ coupon\_value)),}
\NormalTok{                            by }\OtherTok{=}\NormalTok{ .(category, brand\_descr)]}
\NormalTok{brand\_summary[, rank }\SpecialCharTok{:=} \FunctionTok{frankv}\NormalTok{(spend, }\AttributeTok{order =} \SpecialCharTok{{-}}\DecValTok{1}\NormalTok{), by }\OtherTok{=}\NormalTok{ category]}
\end{Highlighting}
\end{Shaded}

\begin{itemize}
\tightlist
\item
  \textbf{Merge the brand ranks in the \texttt{brand\_summary} table
  with the \texttt{purchases} information. Aggregate to the brand level,
  as we did before at the category level. Then calculate per capita
  spending and volume, and normalize the per capita variables to 1.0 in
  2004, as before at the category level. Plot the evolution of brand
  volume for the top four brands, separately for the CSD, diet CSD, and
  bottled water categories.}
\end{itemize}

For bottled water you may add the \texttt{scales\ =\ "free\_y"} option
in \texttt{facet\_wrap}:

\begin{Shaded}
\begin{Highlighting}[]
   \FunctionTok{facet\_wrap}\NormalTok{(..., }\AttributeTok{scales =} \StringTok{"free\_y"}\NormalTok{)}
\end{Highlighting}
\end{Shaded}

By default, the y-axes in a facet wrap are identical across all panels,
but \texttt{free\_y} lets ggplot2 choose different axes for each panel.

\section{Discussion}\label{discussion}

\textbf{Provide a brief discussion of the marketing implications of your
findings.}

\newpage

\section{Appendix: Using a random subsample of the
data}\label{appendix-using-a-random-subsample-of-the-data}

Although not necessary for the analysis in this assignment, especially
as I already created a random subsample of the original data, it is
useful to know how to create such a random subsample.

For example, to draw a random sample of 2 million observations without
replacement, use:

\begin{Shaded}
\begin{Highlighting}[]
\NormalTok{purchases\_sub }\OtherTok{=}\NormalTok{ purchases[}\FunctionTok{sample}\NormalTok{(.N, }\DecValTok{2000000}\NormalTok{)]}
\end{Highlighting}
\end{Shaded}

For details, consult \texttt{?sample}, and note that \texttt{.N} is the
number of rows in a data.table---a variable that the data.table package
automatically supplies.

Alternatively, it may be even better to obtain \emph{all} purchase data
for a random sample of households. First, we obtain a 25 percent sample
of all household codes in the data:

\begin{Shaded}
\begin{Highlighting}[]
\NormalTok{N\_households }\OtherTok{=} \FunctionTok{length}\NormalTok{(}\FunctionTok{unique}\NormalTok{(purchases}\SpecialCharTok{$}\NormalTok{household\_code))}
\NormalTok{N\_subsample  }\OtherTok{=} \FunctionTok{round}\NormalTok{(}\FloatTok{0.25}\SpecialCharTok{*}\NormalTok{N\_households)}

\NormalTok{household\_code\_sub }\OtherTok{=} \FunctionTok{sample}\NormalTok{(}\FunctionTok{unique}\NormalTok{(purchases}\SpecialCharTok{$}\NormalTok{household\_code), N\_subsample)}
\end{Highlighting}
\end{Shaded}

Then extract all data for the chosen household keys. As we already
discussed in the data.table \emph{Keys and Merging} overview, the first
method is more readable yet somewhat slower, the second method is faster
but also more confusing to the novice. The second method also does not
keep its key, so you have to key the \texttt{purchases\_sub\_hh\_a}
data.table later.

\begin{Shaded}
\begin{Highlighting}[]
\NormalTok{purchases\_sub\_hh   }\OtherTok{=}\NormalTok{ purchases[household\_code }\SpecialCharTok{\%in\%}\NormalTok{ household\_code\_sub]}
\NormalTok{purchases\_sub\_hh\_a }\OtherTok{=}\NormalTok{ purchases[.(household\_code\_sub)]}
\end{Highlighting}
\end{Shaded}


\end{document}
